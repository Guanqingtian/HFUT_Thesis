% !TeX root = ./main.tex

\hfutsetup{
  title              = {合肥工业大学学位论文模板示例文档 \hfutthesisversion},
  title*             = {An example of thesis template for \\ Hefei University of Technology \hfutthesisversion},
  stuID              = {2017214563},
  author             = {殷振豪},
  author*            = {Yin Zhenhao},
  speciality         = {数学与应用数学},
  supervisor         = {XXX~教授},
  % date               = {2021-05},  % 默认为当前日期
  secret-level       = {秘密},     % 绝密|机密|秘密|内部交流,注释本行则公开
  secret-year        = {10年},      % 保密年限
  %
  % 数学字体
  % math-style         = GB,  % 可选:GB, TeX, ISO
  math-font          = xits,  % 可选:stix, xits, libertinus
  %
  %
  %
  % Parameters for Non-bachelor's option
  %
  clc               = {分类号}, % 分类号
  %
  %
  %
  % Parameters for master's(docotoral) option 和 答辩委员签名页 参数
  %
  % advisor            = {XXX~教授}, % 用于“专业硕士-中文内封页” 校外导师的填写,默认空                 
  research           = {研究方向}, % 研究方向,默认为 “研究方向”
  signchairman       = {专家工作单位,职称,姓名}, % 答辩委员签名页 主席:  默认为空
  signmember         = {XXX,XXX,XXX \\ XXX,XXX,XXX}, % 答辩委员签名页 委员:  默认为空
  signsupervisor     = {合肥工业大学,XXX,XXX}, % 答辩委员签名页 导师:  默认为空
  %
  %
  %
  % Parameters ONLY for bachelor's option
  %
  ugtype             = {论文}, % 封面第一栏“类型”的选项:设计或者论文,默认“论文”
  hfuteryear         = {2017级}, % 入学年份 
  department         = {应用数学系}, % 系名称
  %
  %
  %
  % Parameters ONLY for doctoral option 和 提名页 参数
  %
  % applydegree        = {}, % 申请学位:理学博士|工学博士, etc. 默认“理学博士”
  % school             = {}, % 默认培养单位:合肥工业大学,难道你想填其他的?
  defensedate        = {2021年6月}, % 答辩时间:格式 “2021年6月”,默认为空
  % phdchairman        = {}, % 提名页 答辩委员会主席:  默认为空
  % phdmember          = {}, % 提名页 评阅人:  默认为空
  % 关于Ph.D的专家清单页,请仔细阅读hftuthesis-doc,详见`chapters/expertspage.tex`
}


% 加载宏包

% 定理类环境宏包
\usepackage{amsthm}

% 插图
\usepackage{graphicx}

% 三线表
\usepackage{booktabs}

% 跨页表格
\usepackage{longtable}

% 算法
\usepackage[ruled,linesnumbered]{algorithm2e}

% SI 量和单位
\usepackage{siunitx}

% 参考文献使用 BibTeX + natbib 宏包
% 顺序编码制
\usepackage[sort]{natbib}
\bibliographystyle{hfutthesis-numerical}

% 著者-出版年制
% \usepackage{natbib}
% \bibliographystyle{hfutthesis-authoryear}

% 本科生参考文献的著录格式
% \usepackage[sort]{natbib}
% \bibliographystyle{hfutthesis-bachelor}

% 参考文献使用 BibLaTeX 宏包
% \usepackage[style=hfutthesis-numeric]{biblatex}
% \usepackage[bibstyle=hfutthesis-numeric,citestyle=hfutthesis-inline]{biblatex}
% \usepackage[style=hfutthesis-authoryear]{biblatex}
% \usepackage[style=hfutthesis-bachelor]{biblatex}
% 声明 BibLaTeX 的数据库
% \addbibresource{bib/ustc.bib}

% 配置图片的默认目录
\graphicspath{{figures/}}

% 数学命令
\makeatletter
\newcommand\dif{%  % 微分符号
  \mathop{}\!%
  \ifhfut@math@style@TeX
    d%
  \else
    \mathrm{d}%
  \fi
}
\makeatother
\newcommand\eu{{\symup{e}}}
\newcommand\iu{{\symup{i}}}

% 用于写文档的命令
\DeclareRobustCommand\cs[1]{\texttt{\char`\\#1}}
\DeclareRobustCommand\pkg{\textsf}
\DeclareRobustCommand\file{\nolinkurl}

% hyperref 宏包在最后调用
\usepackage{hyperref}
